\chapter*{Kurzfassung (Abstract)}
\addcontentsline{toc}{chapter}{Kurzfassung (Abstract)}

Diese Projektarbeit widmet sich der Entwicklung und Implementierung eines auf \ac{KI} basierenden Chatbots bei \ac{FCO}.
Ziel des Projekts ist es, die Effizienz innerhalb der \ac{CIT} Abteilung durch den Einsatz moderner \ac{KI}-Technologien signifikant zu steigern.
Im Rahmen dieser Arbeit wird eine umfassende Analyse gruppenweiter Anwendungsfälle durchgeführt, um spezifische Anforderungen von \ac{FCO} zu identifizieren.

Zur optimalen Umsetzung des Projekts werden verschiedene Large Language Models und Embedding-Modelle auf der SAP \ac{BTP} evaluiert.
Dabei stehen insbesondere die Kriterien Antwortqualität, Antwortzeit und Kosten im Fokus, welche durch die Ergebnisse eine Umfrage unter internen Nutzern gewichtet wurden.
Auf Basis der Analyse erweisen sich das Large Language Model \textit{LLaMA3-70b} und das Embedding-Modell \textit{text-embedding-3-large} als die am besten geeigneten Optionen.

Anschließend erfolgt die Implementierung eines internen Chatbots mithilfe von SAP AI Core, der in der Lage ist, präzise und effiziente Antworten auf spezifische Anfragen zu liefern.
Der Chatbot soll den Arbeitsalltag der Kollegen erleichtern und langfristig einen wertvollen Beitrag zur Verbesserung der internen Prozesse bei \ac{FCO} leisten.
 