\chapter{Einleitung}
\label{einleitung}

\nocite{*}

\section{Historische Entwicklung und technologische Trends}

Seit den 1960er Jahren, als die ersten Automatisierungen repetitiver Produktionsprozesse eingeführt wurden, hat sich die Effizienz in der Industrie signifikant verbessert.
Dieser Trend setzte sich Ende der 1990er Jahre fort, als die Digitalisierung nahezu alle Geschäftsprozesse erfasste.
Zur Jahrtausendwende wurde deutlich, dass die Zukunft jedes Unternehmens in der Nutzung der Informationstechnologie liegt, insbesondere in der Implementierung von Künstlicher Intelligenz (\ac{KI}).
Die innovativsten Unternehmen erkannten frühzeitig die Vorteile dieser neuen Technologien und integrierten sie in ihre Wertschöpfungsketten, um sich einen Wettbewerbsvorteil zu verschaffen. (\cite[S. 406]{Sarferaz2023})

Heutzutage wird geschätzt, dass rund 70 \% aller Unternehmen \ac{KI}-Technologien in ihre Geschäftsprozesse integrieren um die Produktivität zu steigern oder Prozesse vollständig zu automatisieren (\cite[S. 406]{Sarferaz2023}).
Diese Entwicklung unterstreicht die wachsende Bedeutung von \ac{KI} in der modernen Geschäftswelt.

\section{Relevanz für die Freudenberg Gruppe}

Die Freudenberg Gruppe, als global agierendes Unternehmen mit einem breit gefächerten Produkt- und Dienstleistungsportfolio, steht kontinuierlich vor der Herausforderung, sich an die dynamischen Veränderungen im Geschäftsumfeld anzupassen.
Die industrielle Revolution des 18. Jahrhunderts hat gezeigt, wie technologische Fortschritte die Produktionsprozesse radikal verändern können.
Ähnliche transformative Veränderungen sind heute durch die Digitalisierung und Automatisierung zu beobachten, insbesondere durch den Einsatz von Künstlicher Intelligenz. (\cite[S. 405 f.]{Sarferaz2023})

Für die Freudenberg Gruppe ist es daher essenziell, sich intensiv mit den Möglichkeiten und Potenzialen von \ac{KI} auseinanderzusetzen, um die Wettbewerbsfähigkeit zu sichern und zukunftsfähige Geschäftsprozesse zu gestalten.
Der Markt für \ac{KI}-Anwendungen wächst stetig, und Unternehmen, die diese Technologien frühzeitig adaptieren, können erhebliche Vorteile erzielen (\cite[S. 71]{Woo2020}).
Durch die Implementierung einer \ac{KI}-Lösung kann die Freudenberg Gruppe nicht nur ihre internen Abläufe optimieren, sondern auch innovative Ansätze entwickeln, die gruppenweit Anwendung finden können.

\section{Zielsetzung}

Ziel der vorliegenden Arbeit ist die Entwicklung eines Konzepts für einen \ac{KI}-basierten Chatbot, der innerhalb der Freudenberg Gruppe eingesetzt werden soll, 
um die Effizienz interner Arbeitsprozesse durch den Einsatz von \acp{LLM} und Dokumenten-Embeddings signifikant zu steigern. 
Der Fokus liegt hierbei auf der Auswahl und Integration eines leistungsfähigen \ac{LLM}, das in der Lage ist, unternehmensrelevante Informationen auf Grundlage eingebetteter Dokumente präzise und effizient zu extrahieren.

Ein wesentlicher Bestandteil dieser Arbeit ist der systematische Vergleich verschiedener \acp{LLM}, die auf der SAP Business Technology Platform (\ac{BTP}) verfügbar sind. 
Ziel ist es, anhand von zentralen Bewertungskriterien wie Antwortzeit, Antwortqualität und Kosten das Modell zu identifizieren, 
das die besten Voraussetzungen für die Entwicklung eines leistungsstarken Chatbots erfüllt und somit den höchsten Nutzen für die Freudenberg Gruppe generiert.

Der Chatbot soll letztlich dazu befähigt werden, den Mitarbeitern einen schnellen und präzisen Zugriff auf relevante Informationen zu ermöglichen, 
ohne dass eine manuelle Durchsuchung der zugrundeliegenden Dokumente erforderlich ist. Dadurch wird die Informationsbeschaffung optimiert, was eine deutliche Steigerung der Arbeitsproduktivität zur Folge haben soll. 
Die Automatisierung dieser Prozesse trägt maßgeblich zu einer verbesserten Ressourcennutzung und Effizienz im Unternehmen bei.


\section{Struktur der Arbeit}

In Kapitel \ref{grundlagen} werden die theoretischen Grundlagen geschaffen, indem zentrale Konzepte wie Künstliche Intelligenz (\ac{KI}) und deren Unterbereiche, insbesondere Large Language Models (\acp{LLM}), Embeddings und \ac{RAG}, eingeführt werden.
Zusätzlich wird ein Überblick über die SAP Business Technology Platform (\ac{BTP}) sowie die für das Projekt relevante Technologie LlamaIndex gegeben.

Kapitel \ref{analyse} befasst sich mit der Untersuchung der Gründe für die Entscheidung zur Entwicklung eines \ac{KI}-basierten Chatbots innerhalb der Freudenberg Gruppe. 
Basierend auf den spezifischen Anforderungen und Herausforderungen im Unternehmen wird erläutert, warum ein Chatbot die geeignetste \ac{KI}-Lösung darstellt, um die Effizienz der internen Prozesse zu steigern. 

In diesem Zusammenhang wird ein Konzept entwickelt, um die auf der \ac{BTP} verfügbaren \acp{LLM} und Embedding Modelle hinsichtlich ihrer Eignung für die Implementierung eines leistungsfähigen Chatbots zu untersuchen. Dabei wird vorallem auf die spezifischen 
Anforderungen an die Antwortgenauigkeit, Effizienz und Kostenoptimierung eingegangen.

Anschließend wird in Kapitel \ref{evaluation} die Evaluation der verschiedenen \acp{LLM} und Embedding Modelle methodisch durchgeführt. Dabei werden die Modelle anhand definierter Kriterien wie Antwortgenauigkeit, Effizienz und Kosten miteinander verglichen und analysiert. 
Diese Evaluation bildet die Grundlage für die nachfolgende Entwicklung und Implementierung des Chatbots.

In Kapitel \ref{umsetzung} wird die praktische Umsetzung beschrieben. Hierbei werden die evaluierten Modelle implementiert und ein Chatbot auf Basis der vielversprechendsten Modelle entwickelt.
Für die technische Umsetzung wird SAP AI Core verwendet.

Das abschließende Kapitel \ref{fazit} fasst die Ergebnisse der Arbeit zusammen, bewertet die erzielten Erkenntnisse und gibt einen Ausblick auf potenzielle zukünftige Entwicklungen und Erweiterungen des Chatbots.